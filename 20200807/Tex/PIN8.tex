\documentclass[a4paper, 12pt, oneside,openany]{jsarticle}
\usepackage[driver=dvipdfm,truedimen,margin=2cm]{geometry}
\geometry{top=15.4truemm, bottom=15.4truemm, left=15.4truemm, right=15.4truemm}
%\geometry{top=35.1truemm, bottom=30truemm, left=30truemm, right=30truemm}                % See geometry.pdf to learn the layout options. There are lots. 
%\geometry{landscape}                % Activate for for rotated page geometry
%\usepackage[parfill]{parskip}    % Activate to begin paragraphs with an empty line rather than an indent
%\setlength{\oddsidemargin}{-1truecm} %左マージンを2.54cm - 1cm
%\setlength{\topmargin}{-1truecm} %上マージンを2.54cm - 1cm
%\setlength{\headheight}{0mm} %ヘッダーを0mmに
%\setlength{\headsep}{0mm} %ヘッダー本文間の間隔を0に
%\setlength{\textwidth}{55zw}
%\setlength{\textheight}{46\baselineskip}
%\addtolength{\textheight}{\topskip}
\usepackage{graphicx}	%画像を挿入したり,テキストや図の拡大縮小・回転を行うためのパッケージ.
\usepackage{amssymb}	%that provides miscellaneous enhancements for improving the information structure and printed output of documents that contain mathematical formulas. 
\usepackage{epstopdf}	%pstopdf is a Perl script that converts an EPS file to an ‘encapsulated’ PDF file
\usepackage[version=3]{mhchem}	%化学式や反応式をかくためのパッケージ.
%\usepackege{float}	%図の配置に関わるパッケージ
\usepackage[T1]{fontenc} %T1エンコーディングを使用
\usepackage{textcomp}	%TCフォント使用のためのマクロ
\usepackage[utf8]{inputenc}	%ファイルがUTF-8の場合
\renewcommand{\rmdefault}{qtm}	%Timesを本文の欧文フォントに
%\usepackage{lmodern} %Latin Modernフォントを使用
\DeclareGraphicsRule{.tif}{png}{.png}{`convert #1 `dirname #1`/`basename #1 .tif`.png}
\graphicspath{
{../lab_pic/}
}
\title{植物生理ゼミ}
\author{秋田幸太郎}
\begin{document}
	
\section{側根形成過程}
側根形成は、側根形成の位置決定、側根形成の開始、側根の成長、側根の出現の4段階にわけることができます。
図は、側根形成時の根を主なステージごとに示しています。
図の水色はDR5レポーターシステムによって示されたオーキシンの局在を示しています。
図のオレンジは内鞘細胞を示しています。
1の側根形成の位置決定では、側根創始細胞ができることで位置が決まります。
\section{側根形成は内鞘細胞(pericycle pole)から始まる}
内鞘細胞は、図のPで示されるphloem poleと図のXで示されるxylem poleの2種類の細胞群から生じ、その位置関係は図のように直交するような関係です。
側根はその中で、図のXで示されるxylem pole由来の内鞘細胞が分裂することによって生じます。
図の青で示されているのは、GATA23プロモーター下でGUSを発現させたものです。
このGATA23は側根形成開始の指標として知られています。
このGATA23が発現している内鞘細胞から側根創始細胞が形成され、側根が形成されます。
\section{側根形成過程}
2の側根形成の開始段階は、2つの側根創始細胞の非対称分裂が始まる段階です。
図の核移行領域において、
\section{核移行領域では側根創始細胞で核が共通の細胞壁側に移動する}
側根形成を開始する前の側根創始細胞で局所的にオーキシン応答が高まり、核が共通の細胞壁側に移動し極性が確立されます。
そして、中央に2つの小さな細胞、隣接した比較的大きな細胞に非対称に分裂することで側根形成が開始されます。
\section{側根形成過程}
3の側根の成長段階でもオーキシン応答が関わることで細胞分裂と細胞伸長が生じます。
\section{側根原基細胞が内皮、皮層、表皮細胞を押しのけ出現する}
4の側根の出現段階では、成長していく側根原基細胞がその外側に位置する内皮細胞、皮層細胞、表皮細胞を押しのけるように側根が出現します。
\section{PINタンパク質は2種類に分けられる}
PINタンパク質は、図のbの赤矢印で示される親水性ループの長さからロングPINとショートPINに分類されており、その系統樹は図のaで示しています。
ロングPINはPIN1,2,3,4,7の5つが知られ、それぞれ細胞膜に局在し、細胞内のオーキシンを細胞外へ排出する機能をもちます。
ショートPINはPIN5,6,8の3つが知られ、細胞膜またはERに局在し、その局在は発現する細胞の種類により異なり、細胞内のオーキシンの恒常性維持に機能しています。
\section{本研究における目的}
胚発生におけるロングPINの役割は研究が進んでいるもののショートPINの研究は進んでいません。
そこで著者らは、側根形成においてPIN8は機能するのか?機能するのならば、側根形成のどのステージでPIN8は機能するのか?、側根形成においてPIN8はどのような機能を果たすのか?、側根形成においてPIN8はどの細胞で機能するのか?を研究しました。
\section{PIN8は側根形成に機能している}
PIN8が側根形成に関わるのかを調べるために、9日齢の野生型とpin8機能欠損変異体、pin8機能欠損変異体背景にPIN8-GFPを導入した系統の側根を観察しました。
側根1cmあたりの側根の数を相対側根密度として側根形成の指標として用いました。
その結果、野生型と比較しpin8機能欠損変異体の相対側根密度は有意に低下しました。
また、pin8変異体背景にPIN8-GFPを導入した系統の側根密度は野生型と同程度まで回復しました。
つまり、pin8機能欠損変異体の相対側根密度が低下したことは、PIN8が欠失したことによることが確かめられました。

またpin8の機能欠損が主根の成長に影響を与えるのかを確かめるために、野生型、pin8機能欠損変異体、pin8機能欠損変異体背景にPIN8-GFPを導入した系統の主根の長さを計測しました。

その結果は右に示されているグラフで、これら三系統に有意差は確認されませんでした。

これらの結果から、PIN8は主根の成長には機能せず、側根形成を促進していることがわかりました。
では、PIN8は側根形成時のどのステージで機能するのでしょうか?
\section{側根形成過程においてPIN8は側根の出現時(E)に機能する}
主根の長さ1cmあたりの側根原基数を指標にし、野生型とpin8機能欠損変異体で比較しました。
上の写真は側根原基の発生過程のステージを示しており、1-8, eまではグラフの1-8, eと対応しています。
観察の結果、グラフのEのステージ、側根が出現してくるステージにおいて野生型と比較しpin8機能欠損変異体の主根の長さ1cmあたりの側根原基数は有意に低下しました。
\section{ここまでの結果}
この結果からPIN8は側根形成において側根が出現してくる段階で機能することがわかりました。
次に、pin8は根のどこの組織で機能しているのでしょうか。PIN8の発現組織を調べるために、PIN8プロモーター下でGUSを発現する系統を作出し、PIN8の発現組織を調べました。
\section{主根と側根形成時におけるPIN8の発現}
Aは主根を、B-Dは側根形成時の根を示しています。
その結果主根ではGUSの発現は維管束で見られました。
また側根形成時には、Bで示されるように比較的早いステージにおいてはGUSの発現が見られず、C, D以降で示されるステージ8以降においてはGUSの発現が確認されました。
これらの結果から、PIN8は維管束細胞で発現しており、側根形成の比較的遅いステージで発現することが示されました。
\section{ProPIN8:PIN8:GFPは篩管細胞内で発現する}
次にPIN8プロモーター下でPIN8-GFPを発現させPIN8の細胞内局在を確認しました。
図のFのアスタリスクは原生木部の位置を、Pは内鞘細胞を、やじるしはPIN8-GFPを示しています。
Fから、PIN8の細胞内局在は側根形成が始まる原生木部に隣接した内鞘細胞や原生木部ではなく、篩管細胞に局在することがわかりました。
図のG, H, I, Jはメリステム、伸長領域、側根原基、側根出現後のPIN8の局在を示したものです。
PIN8は側根形成の各段階において細胞内で発現していることがわかりました。
\section{ここまでの結果}
これまでの結果からPIN8は側根の出現時に篩管細胞で機能し側根形成を促進することがわかりました。では、PIN8の発現量に依存して側根形成は促進されるのでしょうか?
\section{PIN8の発現量に依存して側根形成が促進される}
PIN8の発現量に依存して側根形成が促進されるのかを検証するために、PIN8プロモーター下でPIN8-GFPを発現する系統を3種類独立に用いて検証しました。

左のグラフは系統1, 3, 4の相対側根密度を示しています。L1に対してL3, L4の相対側根密度は有意に上昇しました。

では、この系統による相対側根密度の差はPIN8の発現量の違いに依存するのかを検証するためにL1, L3, L4のGFP蛍光を観察し定量しました。

真ん中の写真はL1, L3, L4のGFP蛍光を示しており、各系統の相対側根密度に依存して蛍光強度も強いように見られます。

GFP蛍光強度を定量したものが右のグラフです。縦軸にシグナルの相対強度を、横軸に各系統を示しています。
その結果L3, L4はL1に対して有意差が見られ、そのシグナル強度に依存して相対側根密度も高くなっていました。
これらの結果からPIN8の発現に依存して側根形成は促進されることが示されました。
\section{これまでの結果}
ここまでの結果からPIN8が発現する細胞でPIN8の発現量依存的に側根形成が促進されることが示されました。
これらの結果はPIN8が発現する細胞内でオーキシン濃度が高くなることによって側根形成が促進されることを意味するのでしょうか?
\section{PIN8プロモーター下で各遺伝子を発現させるとその局在は変化しなかった}
この可能性を検証するためにPIN8プロモーター下でオーキシン取り込みキャリアであるAUX1, 細胞膜に局在しオーキシンを細胞内から細胞外へ輸送するPIN2, PIN3, PIN8と同じく細胞内に局在するshort PINであるPIN5を発現させ、その側根形成を観察し定量しました。

まずは各系統のPIN8プロモーター下での発現組織をGFP蛍光を指標として観察しました。

その結果AUX1, PIN2, PIN3はPIN8が発現する細胞においても細胞膜に局在し、PIN5は細胞内に局在していることがわかりました。
\section{仮説:PIN8が発現する細胞でPIN8が細胞内オーキシン濃度を上昇させた結果、側根形成が生じる}
PIN8が発現する細胞内でオーキシン濃度が上昇することが側根形成において重要であるのならば、pin8機能欠損変異体背景にオーキシン取り込みキャリアであるAUX1をPIN8プロモーター下で発現させた系統は側根形成が野生型と同定度に回復することが予想されます。

また野生型背景にPIN8プロモーター下でオーキシン排出キャリアであるPIN2, PIN3を発現させた系統では側根形成が野生型よりも低下することが予想されます。
\section{野生型背景においてProPIN8:PIN2 or PIN3を発現させると側根形成は抑制された}
 結果を見てみると予想通り、左のグラフで示されるように野生型背景にPIN8プロモーター下でオーキシン排出キャリアであるPIN2, 3を発現させた系統では相対側根密度は有意に低下しました。
一方でAUX1を発現させた系統の相対側根密度は野生型のそれと比較し有意差は見られませんでした。
\section{pin8機能欠損変異体背景においてProPIN8:AUX1を発現させると側根形成が回復した}
右のグラフから、pin8機能欠損変異体背景にPIN8プロモーター下でオーキシン取り込みキャリアであるAUX1を発現させた系統では相対側根密度が野生型と同程度まで回復しました。
またオーキシン排出キャリアであるPIN2,3を発現させた系統の相対側根密度はpin8機能欠損変異体と同程度で回復は見られませんでした。
これらの結果からPIN8が発現する細胞で細胞内オーキシン濃度が上昇することが側根形成で重要であることが示唆されました。
\section{PIN8が機能する細胞内でPIN5を発現させると側根形成が抑制された}
また野生型背景とpin8機能欠損変異体背景にPIN8プロモーター下でPIN5を発現させると、それぞれ野生型と比較し側根形成は野生型背景では低下、pin8機能欠損変異体背景では回復しませんでした。

この結果から、short PINの中でもPIN5とPIN8はオーキシンの移動に関して異なった働きをすることが示唆されました。
\section{内鞘細胞でPIN8を発現させると側根形成は抑制される}
著者らは側根形成におけるPIN8の機能をさらに調べるために、側根形成の開始に機能し、将来側根創始細胞に分化する内鞘細胞で発現するGATA23のプロモーター下でPIN8を発現する系統を作出し側根形成の様子を確認しました。
その結果、野生型背景にGATA23のプロモーター下でPIN8を発現させた系統では側根形成が抑制されました。
PIN8が発現する細胞においてはPIN8は細胞内オーキシン濃度を上昇させるように機能しているように見えたにもかかわらず、今回はその結果とは逆の結果が得られました。
\section{pGATA23::PINsはいずれも側根形成を抑制した}
野生型と野生型背景にGATA23プロモーター下でPIN1, PIN3, PIN5, PIN8を発現させた系統の側根形成を比較しました。
するといずれの系統の側根形成も野生型と比較し有意に低下しました。
特にPIN8を発現させた系統で顕著に側根形成が抑制されていました。

GATA23プロモーター下ではPIN8がオーキシンを細胞外に排出するように機能していることを示唆しているのでしょうか?
GATA23プロモーター下でこれらのPINにGFPを導入した系統を用いてそのGFP蛍光を観察しました。
その結果、PIN1, 3, 5は通常と同じような局在を示しました。

一方でPIN8はPIN8プロモーター下では細胞内に局在していたのにもかかわらず、GATA23プロモーター下では細胞膜上に局在していました。

この結果はGATA23プロモーター下ではPIN8は細胞膜に局在しオーキシンを細胞内から細胞外へ排出していることを示唆しています。
\section{ここまでのまとめ}
これまでの結果から、PIN8が発現する細胞内でPIN8の発現量に依存して側根形成が促進されたこと。
野生型背景においてPIN8プロモーター下でオーキシン排出キャリアを発現させると側根形成が抑制されたこと。
pin8変異体背景においてPIN8プロモーター下でオーキシン取り込みキャリアを発現させると側根形成が回復したこと。
から、PIN8が発現する細胞内でオーキシン応答が生じることが側根形成で必要であることが示唆されます。
そのため、PIN8プロモーター下でオーキシン応答を阻害する遺伝子を発現させると側根形成は阻害されるはずです。
\section{オーキシン応答の模式図}
図はオーキシン応答の模式図です。
赤はAux/IAAで転写因子であるARFに結合することでオーキシン応答を抑制します。

図の紫と水色で示されているのはARFという転写因子でこれらが機能することでオーキシン応答が生じます。
オーキシン濃度が低いと、左の図のAux/IAAがARFと結合することでARFは抑制され転写調節を行うことができません。

オーキシン濃度が十分高まると右の図で示されるように、Aux/IAAは分解されることでARFが転写調節を行うことができるようになります。
\section{AXR2-1発現時のオーキシン応答の模式図}
そこで著者らはオーキシンリプレッサーであるAux/IAAの機能獲得変異遺伝子であるAXR2-1を用いました。
これは本来ならばオーキシン存在下では分解されるAux/IAAがオーキシン存在下でも分解されない機能を持った変異遺伝子です。
つまり、この遺伝子が発現している細胞ではオーキシンが存在している状態でも常に図で示されるようにオーキシン応答は抑制されてしまいます。
\section{PIN8プロモーター下でAXR2-1を発現させると側根形成は抑制された}
PIN8プロモーター下でAXR2-1-GFPを発現させ細胞内局在を確認しました。
GFP蛍光はPIN8が発現している篩管細胞で発現していることが示されました。

右のグラフから、PIN8プロモーター下でAXR2-1を発現した系統は野生型と比較し有意に側根形成が抑制されました。
この結果から、PIN8が発現する篩管細胞でオーキシン応答が生じることが側根形成に必要であることが示されました。
\section{pin8機能欠損変異体ではオーキシン関連遺伝子、側根形成に関わる遺伝子の発現が有意に低下した}
次にPIN8が側根形成に関連したオーキシン信号伝達に影響を与えるのかを調べるために、野生型とpin8機能欠損変異体のオーキシン信号伝達と側根形成に関連した遺伝子の発現解析を行いました。

pin8機能欠損変異体ではGATA23遺伝子やオーキシン取り込みキャリアであるLAX3、側根形成に機能するLBD18, 29、細胞壁の緩みに機能するエクスパンシンEXPA14, 17の発現が野生型と比較し有意に低下しました。
この結果も、PIN8がオーキシン応答を引き起こすために重要であることを支持しています。
\section{本研究のまとめ}
本研究から、側根形成時にPIN8は側根の出現時に主に機能し、細胞内のオーキシンの恒常性を維持し、篩管細胞で発現し、そこでオーキシン応答が生じることが側根形成に必要であることが示されました。
\bibliographystyle{pnas2009}
\bibliography{卒業研究}
\end{document}
