\documentclass[a4paper, 12pt, oneside,openany]{jsarticle}
\usepackage[driver=dvipdfm,truedimen,margin=2cm]{geometry}
\geometry{top=15.4truemm, bottom=15.4truemm, left=15.4truemm, right=15.4truemm}
%\geometry{top=35.1truemm, bottom=30truemm, left=30truemm, right=30truemm}                % See geometry.pdf to learn the layout options. There are lots. 
%\geometry{landscape}                % Activate for for rotated page geometry
%\usepackage[parfill]{parskip}    % Activate to begin paragraphs with an empty line rather than an indent
%\setlength{\oddsidemargin}{-1truecm} %左マージンを2.54cm - 1cm
%\setlength{\topmargin}{-1truecm} %上マージンを2.54cm - 1cm
%\setlength{\headheight}{0mm} %ヘッダーを0mmに
%\setlength{\headsep}{0mm} %ヘッダー本文間の間隔を0に
%\setlength{\textwidth}{55zw}
%\setlength{\textheight}{46\baselineskip}
%\addtolength{\textheight}{\topskip}
\usepackage{graphicx}	%画像を挿入したり,テキストや図の拡大縮小・回転を行うためのパッケージ.
\usepackage{amssymb}	%that provides miscellaneous enhancements for improving the information structure and printed output of documents that contain mathematical formulas. 
\usepackage{epstopdf}	%pstopdf is a Perl script that converts an EPS file to an ‘encapsulated’ PDF file
\usepackage[version=3]{mhchem}	%化学式や反応式をかくためのパッケージ.
%\usepackege{float}	%図の配置に関わるパッケージ
\usepackage[T1]{fontenc} %T1エンコーディングを使用
\usepackage{textcomp}	%TCフォント使用のためのマクロ
\usepackage[utf8]{inputenc}	%ファイルがUTF-8の場合
\renewcommand{\rmdefault}{qtm}	%Timesを本文の欧文フォントに
%\usepackage{lmodern} %Latin Modernフォントを使用
\DeclareGraphicsRule{.tif}{png}{.png}{`convert #1 `dirname #1`/`basename #1 .tif`.png}
\graphicspath{
{../lab_pic/}
}
\title{植物生理ゼミ}
\author{秋田幸太郎}
\begin{document}
	
窒素は植物にとって必須のマクロ栄養素で、核酸やタンパク質の合成に必要不可欠です。
自然環境の土壌における窒素源はアンモニウムイオンと硝酸イオンです、
しかし、土壌中のほとんどの窒素源は硝酸イオンです。
土壌中の硝酸イオンの分布は均一ではなく、土壌のタイプや気候により異なっています。
また一般的に土壌中の硝酸イオン濃度は低いため、植物は外部の窒素利用可能性と植物体内の窒素状況に応じて適切に根から窒素を取り込む仕組みを発達させてきたと考えられています。
土壌中の硝酸イオンはNRT1, 2, 3という硝酸イオントランスポーターによって細胞に取り込まれます。
これらの硝酸イオントランスポーターは高親和性と低親和性の輸送システムに分類され、外部環境の硝酸イオン濃度がそれぞれ低い時と高い時に硝酸イオン取り込み活性をもつことが知られています。
高親和性の硝酸イオントランスポーターとしてNRT2, 3がよく知られています。
一方、NRT1は低親和性の硝酸イオントランスポーターとして知られています。

取り込まれた硝酸イオンは表皮細胞から皮層、内皮細胞まではシンプラストやアポプラストを介して移動し、内鞘細胞へはシンプラストを介して移動します。
内鞘細胞からNRT1.5を介して木部へ積み込まれます。
このように硝酸イオンを取り込み木部に積み込むことで全身へ硝酸イオンを運びます。

根において窒素飢餓になると窒素要求のシグナルとして根からシュートへCEPというペプチドが輸送されます。
このCEPはシュートでCEPR1というレセプターで受容されます。
CEPR1の下流でCEPD1/2というポリペプチドが合成されシュートから根へ輸送されます。
このCEPD1/2が窒素飢餓になっていない側の根で最終的にNRT2.1などの高親和性硝酸イオントランスポーター遺伝子の発現を誘導します。
このように、局所的な根の窒素状態はシュートで統合され、シグナルの元となった場所とは異なる根で硝酸イオンの効率的な取り込みを促進することで窒素飢餓に対応します。

このように根の窒素状態の統合に関する分子機構は理解が進んできました。
しかし、シュートの窒素状態を統合し、それに対する全身的な窒素獲得応答に関連する分子機構は理解が進んでいません。
本研究では、このシュートの窒素状態を統合し根に伝達することで窒素取り込みを促進する分子を探索することを目的に研究が行われました。

はじめに、CEPD1,2のアミノ酸配列のアライメント解析を行った結果からCEPD1/2と類似したCEPDL1, 2という2つのペプチドが同定されました。
これら2つのペプチドは102アミノ酸から成りCEPD1のアミノ酸配列の類似度は80\%以上と非常に類似していたことから窒素状態を伝達する機能を有している可能性があります。
これら2つのペプチド, CEPDL1/2をターゲットにし研究を進めていきました。

CEPDL1, CEPDL2過剰発現体を作出しアンモニウムイオンと硝酸イオン存在下と、硝酸イオン存在下におけるそれらのNRT2.1の発現量を比較しました。
通常NRT2.1の発現量はアンモニウムイオン存在下では抑制されます。
しかしアンモニウムイオン存在下でもCEPDL2過剰発現系統では野生型と比較し25倍もNRT2.1の発現量が上昇しました。
また、アンモニウムイオン非存在下で硝酸イオンのみ存在する状態でもCEPDL2過剰発現系統のNRT2.1発現量は野生型と比較し有意に上昇しました。
これらの結果からCEPDL2は窒素獲得の経路で重要な役割を果たしていることが示唆されました。
そのため、著者らはCEPDL2に着目しさらに研究を進めました。

まずはCEPDL2遺伝子がどこで発現しているかを10日齢のCEPDL2ネイティブプロモーター下でGUSを発現する系統を用いて調べました。
するとCEPDL2遺伝子は子葉と成熟した葉の師部細胞に発現していることがわかりました。
また7日齢の野生型のシュートと根でCEPDL2遺伝子の発現解析を行った結果CEPDL2遺伝子はシュートで特異的に発現していることがわかりました。

窒素獲得に関してCEPDL2がどのように機能するのかをさらに調べるために、硝酸イオントランスポーター遺伝子とアンモニウムイオントランスポーター遺伝子の発現量を定量PCRで調べました。
その結果が図で青が野生型を赤がCEPDL2過剰発現系統を示しています。
NRT1.5を除くNRTは硝酸イオンを高い特異性で取り込む硝酸イオントランスポーターです。
CEPDL2過剰発現系統では、これらの遺伝子の発現量が野生型と比較し増加していました。

一方、CEPDL2過剰発現系統では、野生型と比較しアンモニウムイオントランスポーターの発現量の増加は見られませんでした。

CEPDL2の機能を調べるために、硝酸イオン濃度を変えた培地を用いて硝酸イオンをどの程度取り込むかを調べました。
CEPDL2過剰発現系統は硝酸イオンが低濃度の場合でも高濃度の場合でも野生型と比較し有意に硝酸イオンを取り込んでいることがわかりました。

次に野生型とCEPDL2過剰発現系統のシュートと根での窒素含有量を調べました。
CEPDL2過剰発現系統のシュートの窒素含有量は野生型と比較し1.4倍と高くなっていました。
一方でCEPDL2過剰発現系統の根の窒素含有量は野生型と比較し47\% と低下していました。
この結果は先ほど示したNRT1.5の発現量が上昇していたことから解釈できます。
NRT1.5は根から木部への窒素積み込みを担うタンパク質として知られているからです。

CEPDL2が根からの窒素濃度に依存したシグナルによって発現が上昇するのか、それともシュート自身の窒素状態によって発現が生じるのかを調べました。

シロイヌナズナをそれぞれ10mMのアンモニウムイオン、硝酸イオンを含んだ培地で12日間育てました。
その後シロイヌナズナの根とシュートを切り離し、シュートをそれぞれ窒素源が豊富にある対照区の培地とアンモニウムイオンと硝酸イオンが含まれない実験区の培地に移し, 24時間培養した葉をサンプルとし発現解析に用いました。
このように根とシュートを切り離すことで、根由来のシグナルの影響を除去しシュート自体の窒素状態によってCEPD1/2, CEPDL1/2の発現が変動するかを調べました。

発現解析の結果が図のグラフです。
cepr1-1のCEPD1, CEPD2, CEPDL1の発現量は野生型と比較し有意差はありませんでした。

一方、CEPDL2の発現量はコントロールと比較した時に野生型においてもcepr1-1においても有意に発現量が上昇していました。
これらの結果からCEPDL2の発現は根からのシグナルからではなく、シュートの窒素量によって制御されることがわかりました。

CEPDL2の発現量は窒素が含まれていない培地に移してからどのように変動していくのでしょうか。
野生型を窒素が含まれていない培地に移し替え、各時間におけるCEPDL2の発現量を調べました。
その結果、CEPDL2の発現量は6から12時間の間に上昇していました。

また同じように野生型を窒素が含まれていない培地に移し替え、各時間における窒素含有量を調べました。
その結果は右図で示されるように、実験区におけるシュートの窒素量変化はCEPDL2の発現量が増加した6から12時間の間に低下しました。

CEPDL2の発現量がシュートの窒素含有量が低下した時に上昇するのか調べるために実験を続けました。
14日齢と21日齢の野生型を10mM, 3mM, 1mMの硝酸イオンを含む培地で培養し、それらのシュートにおけるCEPDL2の発現量と窒素含有量を調べました。
左の図から1mMという低濃度の硝酸イオンを含む培地で育てるとCEPDL2の発現量は上昇することがわかりました。
また右の図で示されるように1mMの培地で育てた野生型ではシュートの窒素量が低下しています。
これらの結果は先ほど示したシュートの窒素量が低下するとCEPDL2の発現が上昇した結果と一致します。

次にcepdl2の機能をさらに調べるために、cepdl2機能欠損変異体を作出し研究に用いました。
cepdl2-1変異体は野生型よりも小さく生育が悪い表現型を示しています。

野生型とcepdl2-1機能欠損変異体を10mM, 3mM, 1mMの硝酸イオンを含む培地で14日間または21日間育てた後に、シュートの生重量を測定しました。
その結果、21日齢のcepdl2機能欠損変異体の生重量は野生型のそれと比較し有意に低下しました。

さらにcepdl2機能欠損変異体の硝酸イオン濃度が低濃度の場合に硝酸イオンを取り込む能力は野生型と比較し有意に低下しました。
この結果もcepdl2-1機能欠損変異体の表現型と一致しました。

さらに野生型とcepdl2-1機能欠損変異体のシュートと根の窒素含有量を調べました。
すると、cepdl2-1機能欠損変異体のシュートの窒素含有量は野生型のそれと比較し有意に低下しました。

しかしcepdl2-1機能欠損変異体の根の窒素含有量は野生型のそれと比較し有意に上昇していました。

続いて、3mMの硝酸イオンを含む培地で育った21日齢の野生型とcepdl2-1機能欠損変異体のNRT2.1, NRT1.5, NRT3.1の発現量を調べました。
すると、高親和性硝酸イオントランスポーターであるNRT2.1遺伝子の発現量は野生型のそれと比較し有意に低下しました。

また根に取り込んだ硝酸イオンを師部細胞に積み込む機能を持つNRT1.5の発現量も野生型と比較し有意に低下しました。

これらの結果から、CEPDL2は高親和性の窒素取り込みと窒素の根からシュートへの取り込みを正に制御することが確かめられました。

また、図eが示すNRT1.5の発現量の低下が図のdのcepdl2-1変異体の根の窒素含有量が野生型と比較し有意に増加したことの解釈として考えられます。

CEPDL2がシュートから根に移動するのかを検証するために、CEPDL2のネイティブプロモーター下でCEPDL2の5’側にGFPをつけて野生型に導入し根におけるGFP蛍光を観察しました。
するとGFP蛍光は主に師部細胞に強く見られ、内皮、皮層、表皮細胞にも確認されました。

さらにGFP-CEPDL2と野生型を接木しGFP蛍光を確認しました。
その結果、一番左の図のようにシュート側、根側を共に野生型にした際にはGFP蛍光は確認できませんでした。
左から二番目の図のように、シュート側をGFP-CEPDL2、根側を野生型にし接木したところ根でGFP蛍光が確認されました。
左から三番目、シュート側を野生型、根側をGFP-CEPDL2導入系統にしたところGFP蛍光は確認されませんでした。
これらの結果からCEPDL2はシュートから根に輸送されることが確認されました。

CEPDL2の機能をさらに調べるためにcepd1, 2, cepdl2三重変異体を作出し実験に用いました。
はじめに三重変異体の高親和性硝酸イオン取り込み活性を調べました。
cepd1,2 cepdl2三重変異体の窒素取り込み能力は野生型と比較し有意に低下しました。

さらにcepd1 2 cepdl2三重変異体では硝酸イオン取り込みに機能するNRT2.1, NRT3.1の発現量も低下していました。

野生型とcepd1 2 cepdl2三重変異体とcepd1 2 cepdl2三重変異体にGFP-CEPDL2を導入しCEPDL2を相補させた系統を用いて接木実験を行いました。
Wは野生型をtはcepd1 2 cepdl2三重変異体を、Cは cepd1 2 cepdl2三重変異体にGFP-CEPDL2を導入しCEPDL2を相補させた系統を示しています。
上側がシュート側を下側が根側を示しています。
シュート側を、CEPDL2を相補させた系統の窒素取り込み能力はシュート側、根側ともに野生型のコントロールと同程度を示しました。

一方でシュート側を三重変異体にした場合はHATS活性がシュート側、根側ともに野生型のものと比較し有意に低下しました。
この結果から、CEPDL2がシュートで発現することが根での高親和性の窒素吸収に重要であることがわかりました。

左の図は野生型を右の図はcepd1,2二重変異体のCEP経路とCEPDL2経路を示しています。
cepdl1,2二重変異体ではCEP経路を介した硝酸イオントランスポーター遺伝子の発現誘導ができません。
野生型とcepdl1,2二重変異体を様々な硝酸イオン濃度で培養した時に、cepd1,2二重変異体では硝酸イオンを取り込むためにCEPDL2の発現量が上昇しているか検証しました。

その結果をグラフで示しています。
硝酸イオン濃度10mM, 3mMで培養した時はcepd1,2二重変異体のCEPDL2発現量は同じ濃度で培養した野生型のそれと比較し有意に上昇しました。
また硝酸イオン濃度を1mMで培養した際は野生型のCEPDL2発現量は10mMで培養した時と比較し10倍近くになり、cepd1,2二重変異体のCEPDL2発現量とは有意差は見られませんでした。
この結果からCEPDL2はCEPD1,2に対して補償的に機能することがわかりました。

本研究からCEPDL2がシュートの窒素状態を根に伝達し、硝酸イオン取り込みを促進することがわかりました。
窒素が十分にある環境、中間くらいの環境と低濃度の環境に分けてどのようにCEPD1,2とCEPDL2が機能するのかを示したのが図です。
左が窒素が十分にある環境を示しており、窒素供給が窒素需要と一致している場合はCEPD1,2とCEPDL2の発現量は基準レベルに維持されます。
中央の図はシュートの窒素要求量に対して窒素供給が不足している場合を示しています。
その場合はCEPD1,2の発現量は基準値くらいでもCEPDL2の発現量は増加し窒素吸収を促進するように機能します。
右の図は根が窒素が限られた環境に晒され、根での窒素要求量に対して窒素吸収が不十分な場合を示しています。
その場合はCEPD1,2経路とCEPDL2経路が正に制御されることで窒素吸収を促進するように機能します。
\bibliographystyle{pnas2009}
\bibliography{卒業研究}
\end{document}
